\documentclass[10pt,a4paper]{letter}
\usepackage[utf8]{inputenc}
\usepackage[french]{babel}
\usepackage[T1]{fontenc}
\usepackage{amsmath}
\usepackage{amsfonts}
\usepackage{amssymb}
\usepackage[left=1cm,right=1cm,top=1cm,bottom=1cm]{geometry}

\newenvironment{exercise}{}{}
\renewcommand{\title}[1]{#1}
\renewcommand{\date}[1]{#1}
\newcommand{\content}[1]{#1}
\newcommand{\solution}[1]{#1}



\begin{document}

\begin{exercise}
\title{Exercice 1}
\date{2025-02-14}
\content{test}
\solution{Solution de l'exercice 2.}
\end{exercise}

\begin{exercise}
\title{Exercice 2}
\date{2025-02-15}
\content{
Soit $f$ une fonction telle que 
$$
f(x)=3x+2
$$
Combien de réels $a$ existe-t-il tel que $f(a)=3$?
}
\solution{Il en exise un seul :

$$
f(x)=3\Leftrightarrow 3x+2=3\Leftrightarrow 3x=1\Leftrightarrow x=\frac{1}{3}
$$
}
\end{exercise}
    

\end{document}

